% Created 2021-09-24 ven. 10:46
% Intended LaTeX compiler: pdflatex
\documentclass[11pt]{article}
\usepackage[utf8]{inputenc}
\usepackage[T1]{fontenc}
\usepackage{graphicx}
\usepackage{grffile}
\usepackage{longtable}
\usepackage{wrapfig}
\usepackage{rotating}
\usepackage[normalem]{ulem}
\usepackage{amsmath}
\usepackage{textcomp}
\usepackage{amssymb}
\usepackage{capt-of}
\usepackage{hyperref}
\date{\today}
\title{TP1}
\hypersetup{
 pdfauthor={msi},
 pdftitle={TP1},
 pdfkeywords={},
 pdfsubject={},
 pdfcreator={Emacs 27.1 (Org mode 9.4.6)}, 
 pdflang={English}}
\begin{document}

\maketitle
\tableofcontents

\section{Rappel rapide des méthodes}
\label{sec:orga81117c}
\subsection{Méthode de Gauss}
\label{sec:orgb6fa052}
Cette méthode permet de trouver une solution exacte au système \(Ax = b\) en un nombre fini d'étape.


Pour ce faire, cette méthode se fait en plusieurs étapes :

\begin{enumerate}
\item La triangularisation
On doit passer du système \(Ax=b\) au système \(A'x=b'\) où A' est une matrice triangulaire supérieure. L'algorithme utilisé est disponible dans le programme.
\item La résolution facile
\texttt{Nécessite aucun 0 sur la diagonale de A}
\end{enumerate}

\subsection{Méthode de Jacobi}
\label{sec:orgef40923}

Cette méthode fait partie des méthodes itératives, où l'on cherche à se rapprocher, avec une suite d'itération définie, à une solution exacte.

Pour cette méthode, nous devons tout d'abord décomposer A sous la forme A = D -E -F

\begin{enumerate}
\item D est la matrice nul de taille A, sauf sur sa diagonale où D possède les coefficient de A.
\item -E est la matrice triangulaire inférieure de A
\item -F est la matrice triangulaire supérieur de A
\end{enumerate}



De plus, on pose \(M = D\) et \(N = E + F\)

On obtient donc le système : 

\[Ax = b \Longleftrightarrow Dx^{k+1} = (E + F)x^k + b \]

pour l'itération \(k+1\)

De plus, l'algorithme de Jacobi s'écrit avec une précision \(\epsilon\) : 
\end{document}