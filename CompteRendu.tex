% Created 2021-09-30 jeu. 18:15
% Intended LaTeX compiler: pdflatex
\documentclass[letter]{article}
                      \usepackage[utf8]{inputenc}
\usepackage[T1]{fontenc}
\usepackage{graphicx}
\usepackage{grffile}
\usepackage{longtable}
\usepackage{wrapfig}
\usepackage{rotating}
\usepackage[normalem]{ulem}
\usepackage{amsmath}
\usepackage{textcomp}
\usepackage{amssymb}
\usepackage{capt-of}
\usepackage{hyperref}
\usepackage[a4paper,left=2cm,right=2cm,top=2cm,bottom=2cm]{geometry}
\usepackage[frenchb, ]{babel}
\usepackage{libertine}
\usepackage[pdftex]{graphicx}
\setlength{\parskip}{1ex plus 0.5ex minus 0.2ex}
\newcommand{\hsp}{\hspace{20pt}}
\newcommand{\HRule}{\rule{\linewidth}{0.5mm}}
\date{\today}
\title{Résolution de systèmes linéaires : Méthodes directes}
\begin{document}


%chargement de la page de garde
\input{/home/msi/Documents/Org/Latex/Setupfile/Pagedegarde/Pagedegarde1/pagedegarde1.org}


\setcounter{tocdepth}{2}
\tableofcontents

\newpage




\section{Rappel rapide des méthodes}
\label{sec:org7ef68bf}
\subsection{Méthode de Gauss}
\label{sec:org29b02a0}
Cette méthode permet de trouver une solution exacte au système \(Ax = b\) en un nombre fini d'étape.


Pour ce faire, cette méthode se fait en plusieurs étapes :

\begin{enumerate}
\item La triangularisation
On doit passer du système \(Ax=b\) au système \(A'x=b'\) où A' est une matrice triangulaire supérieure. L'algorithme utilisé est disponible dans le programme.
\item La résolution facile
\texttt{Nécessite aucun 0 sur la diagonale de A}
\end{enumerate}

\subsection{Méthode de Jacobi}
\label{sec:org2184769}

Cette méthode fait partie des méthodes itératives, où l'on cherche à se rapprocher, avec une suite d'itération définie, à une solution exacte.

Pour cette méthode, nous devons tout d'abord décomposer A sous la forme A = D -E -F

\begin{enumerate}
\item D est la matrice nul de taille A, sauf sur sa diagonale où D possède les coefficients de A.
\item -E est la matrice triangulaire inférieure de A
\item -F est la matrice triangulaire supérieur de A
\end{enumerate}



De plus, on pose \(M = D\) et \(N = E + F\)

On obtient donc le système : 

\[Ax = b \Longleftrightarrow Dx^{k+1} = (E + F)x^k + b \]

pour l'itération \(k+1\)

De plus, l'algorithme de Jacobi s'écrit avec une précision \(\epsilon\) : 




\section{Présentation des programmes commentés}
\label{sec:orgfad8b0b}

\subsection{Présentation Général :}
\label{sec:orga078d22}

\subsubsection{Les différents fichiers utilisés}
\label{sec:org1102491}
Pour effectuer ce travail, nous avons décidé de séparer notre programme en plusieurs fichiers : 

\begin{enumerate}
\item main.c, qui est notre fichier appelant les divers fonctions présentent dans
\item fonction.c, puis
\item fonction.h, permettant de définir les différentes structures et les headers des fonctions, et enfin
\item main.h, où les différentes bibliothèques sont déclarées
\item De plus, il y a un Makefile, qui nous permet de compiler et tester notre programme efficacement
\end{enumerate}







\subsubsection{Les structures ainsi que les fonctions usuelles}
\label{sec:orge2bae7b}

\begin{enumerate}
\item La structure de nos matrice
\label{sec:org9733350}

Nous avons choisis de définir notre structure matrice de la sorte :

\begin{verbatim}

  typedef struct matrice
{
  int longueur;
  int largeur;
  long double **Mat;
} matrice;

\end{verbatim}

Mettre la longueur et la largeur de la matrice directement dans la structure permet à nos fonction de ne plus les avoir en paramètre. De plus, les matrices sont des doubles tableaux. On utilise dont un pointeur de pointeur pour permettre, grâce à la fonction 


\item Les fonctions usuelles
\label{sec:org1fd1b86}


\begin{enumerate}
\item \emph{creerMatrice}, qui prend un paramètre la largeur et la longueur, et qui renvoie un pointeur de matrice. Cette fonction fait principalement la création et l'initialisation d'un tableau 2D avec des 0.
\begin{verbatim}

matrice *creerMatrice(int largeur, int longueur)
{
  /* création d'un pointeur sur une structure matrice (avec l'espace
     alloué de la structure matrice)  */
  matrice *fini = (matrice *)malloc(sizeof(matrice));

  /* attribution de la longueur et de la largeur */
  fini->longueur = longueur;
  fini->largeur = largeur;

  /* Création du double tableau, la matrice même (initialisé à 0)  */
  fini->Mat = (long double **)malloc(fini->longueur * sizeof(long double *));
  /* parcourt de la matrice avec la double boucle for */
  for (int i = 0; i < fini->longueur; i++)
    {
      fini->Mat[i] = (long double *)malloc(fini->largeur * sizeof(long double));
      for (int j = 0; j < fini->largeur; j++)
	{
	  fini->Mat[i][j] = 0;
	}
    }

  /* renvoie de la fini */
  return fini;
}     

\end{verbatim}

\item \emph{destroyMatrice}, qui permet de supprimer et de vider la mémoire d'une matrice.
\begin{verbatim}
void destroyMatrice(matrice *mat)
{
  for (int i = 0; i < mat->longueur; i++)
  {
    free(mat->Mat[i]);
  }
  free(mat->Mat);
  mat->longueur = 0;
  mat->largeur = 0;
  free(mat);
}
\end{verbatim}

\item / afficheMatrice/, qui permet tout simplement d'afficher une matrice.

\item \emph{remplisAleaBcpZero}, qui permet de remplir une matrice avec environ 70\% de 0.

\item \emph{remplisAlea}, permettant de remplir une matrice avec des nombres aléatoire (etre -100 et 100)

\item \emph{remplisAleaInt}, déclinaison de remplisAlea avec des entier

\item \emph{additionMatrice}, qui, comme son nom l'indique, d'additionner 2 matrices
\begin{verbatim}
matrice *additionMatrice(matrice mat1, matrice mat2)
{
  // initialisation des variables
  matrice *fini;
  long double res;

  // initialisation de la matrice de retour

  // création de la matrice fini avec la taille de la taille max entre
  // 2 matrices
  fini = creerMatrice(
      (mat1.largeur > mat2.largeur) ? mat1.largeur : mat2.largeur,
      (mat1.longueur > mat2.longueur) ? mat1.longueur : mat2.longueur);

  // mise du resultat dans la matrice de retour
  for (int i = 0; i < fini->longueur; i++)
  {
    for (int j = 0; j < fini->largeur; j++)
    {
      res = 0;
      if ((mat1.longueur > i) && (mat1.largeur > j))
      {
	res += mat1.Mat[i][j];
      }
      if ((mat2.longueur > i) && (mat2.largeur > j))
      {
	res += mat2.Mat[i][j];
      }
      fini->Mat[i][j] = res;
    }
  }

  // retour de la matrice de resultat
  return fini;
}
\end{verbatim}

\item \emph{soustractino}, qui permet de les soustraire

\item \emph{multiplicationMatrice}, qui les multiplie
\begin{verbatim}
matrice *multiplicationMatrice(matrice mat1, matrice mat2)
{
  // verification des conditions
  if (mat1.largeur != mat2.longueur)
  {
    printf("On ne peu pas multiplier ces deux matrices ensemble.\n");
    return NULL;
  }

  // initialisation des variables
  matrice *Xfini = creerMatrice(mat2.largeur, mat1.longueur);
  long double lambda;
  /* afficheMatrice(mat1); */
  // calcule de chaque case une par une
  for (int i = 0; i < mat1.longueur; i++)
  {
    for (int j = 0; j < mat2.largeur; j++)
    {
      lambda = 0;
      for (int h = 0; h < mat1.largeur; h++)
      {
	lambda += (mat1.Mat[i][h] * mat2.Mat[h][j]);
      }
      Xfini->Mat[i][j] = lambda;
    }
  }

  return Xfini;
}
\end{verbatim}
\end{enumerate}
\end{enumerate}






\subsection{Gauss}
\label{sec:org2b49530}

\subsection{Jacobi}
\label{sec:org68185ca}

\subsubsection{Gestion des matrices non diagonales dominantes}
\label{sec:orgd8bdd70}


Tout d'abord, pour utiliser la méthode de Jacobi, il faut que les matrices soient de la bonne taille. On vérifie donc que la matrice A est carré, que la longueur de A est égal à celle de B, ainsi que la largeur de B qui doit être égal à un. Ces conditions sont testés avec ce code :
\begin{verbatim}

/* gestion des cas d'erreur pouvant faire echouer la methode jacobi*/
if ((A->largeur != A->longueur) || (A->longueur != B->longueur) ||
    (B->largeur != 1))
{
  printf(
      "Les matrice ne sont pas de la taille nécessaire a leurs résolution.");
  return B;
}
 else


\end{verbatim}



Puis, il faut aussi que les matrices soient strictement diagonales dominantes.
Pour ce faire, nous pouvons mettre au début de la fonction "Jacobi", une double boucle for qui test si les diagonales sont bien dominantes :

\begin{verbatim}

  else
{
  for (int i = 0; i < A->longueur; i++)
  {
    int verifieur = 0;
    for (int j = 0; j < A->longueur; j++)
    {
      if (j != i)
      {
	verifieur += fabsl(A->Mat[i][j]);
      }
    }
    if (verifieur > A->Mat[i][i])
    {
      printf("La matrice n'est pas à diagonale dominante et ne vas donc pas "
	     "converger...\n");
      return B;
    }
  }
}


\end{verbatim}


Sinon, le reste du code est exécuté normalement.








\subsubsection{Méthode général}
\label{sec:org5f5c05a}

Suites à la création des matrices x, D, E, F (et N) à l'aide des fonction usuelles, il faut les initialiser avec les bonnes valeurs. On a choisit de faire un parcours de la matrice A, et lorsque les conditions sont réunies, nous entrons la valeur de A en fonction de la matrice.

\begin{verbatim}
  /* initialisation de D E et F (et N)*/
for (int i = 0; i < A->longueur; i++)
{
  for (int j = 0; j < A->largeur; j++)
  {
    if (i == j)
    {
      D->Mat[i][j] = A->Mat[i][j];
      E->Mat[i][j] = 0;
      F->Mat[i][j] = 0;
    }
    else if (i < j)
    {
      D->Mat[i][j] = 0;
      E->Mat[i][j] = -(A->Mat[i][j]);
      F->Mat[i][j] = 0;
    }
    else
    {
      D->Mat[i][j] = 0;
      E->Mat[i][j] = 0;
      F->Mat[i][j] = -(A->Mat[i][j]);
    }
    N->Mat[i][j] = E->Mat[i][j] + F->Mat[i][j];
  }
}


\end{verbatim}

Puis, on initialise la marge d'erreur. Nous inversons également D (avec la fonction \emph{InversematriceD}), car il faut utiliser \(D^{-1}\)


\begin{verbatim}

void InversematriceD(int taille, matrice *D)
{
  for (int i = 0; i < D->longueur; i++)
    {
      for (int j = 0; j < D->largeur; j++)
	{
	  if ((i == j) && (D->Mat[i][j] != 0))
	    {
	      D->Mat[i][j] = 1 / D->Mat[i][j];
	    }
	}
    }
}


float erreur = Eps + 1;
InversematriceD(D->longueur, D);


\end{verbatim}

Enfin, nous programmons la méthode de Jacobi grâce aux fonctions "multiplicationMatrice, additionMatrice".
La boucle utiliser est un tant que, car nous ne savons pas quand la marge d'erreur sera respecté pour sortir de la boucle while.
La variable erreur doit également être mise à jour à chaque passage dans la boucle, en utilisant la fonction \emph{Norme}.



\begin{verbatim}


float Norme(matrice *colonne)
{
  float norme = 0;
  for (int i = 0; i < colonne->longueur; ++i)
    {
      norme = norme + pow(colonne->Mat[i][0], 2);
    }
  return sqrt(norme);
}

while (erreur > Eps)
  {
    // nouvelle valeur de x selon la formule
    x = multiplicationMatrice(*D, *additionMatrice(*(multiplicationMatrice(*N, *x)), *B));
    // nouvelle valeur d'erreur selon la formule
    erreur = Norme(soustractino(*multiplicationMatrice(*A, *x), *B));

  }
return x;
}

\end{verbatim}

\subsection{Programme final}
\label{sec:org47dda42}

Les différentes fonctions sont appelées au fur et à mesure du main.c, en laissant le choix à l'utilisateur avec quelle matrice il veut faire ses tests, ainsi que la méthode à utiliser. Les différents choix sont regroupés dans un switch.

\section{Présentation des matrices test au dos de la feuille}
\label{sec:orgb2cb275}

\href{fonction.c}{aller ici}

\section{Conclusion ssur les méthodes}
\label{sec:org1717882}

\subsection{{\bfseries\sffamily TODO} Comparaison}
\label{sec:org34dd967}

\subsection{{\bfseries\sffamily TODO} Cadre d'utilisation}
\label{sec:orgb485f48}

\subsection{{\bfseries\sffamily TODO} Stabilité}
\label{sec:org1547ee7}
Cette méthode a un coût de l'ordre de 3n2+2n par itération. Elle converge moins vite que la méthode de Gauss-Seidel, mais est très facilement parallélisable. 
\end{document}